\documentclass{assignment}
\usepackage[pdftex]{graphicx}
\usepackage{xcolor}
\definecolor{LightGray}{gray}{0.95}
\usepackage{fancyvrb, minted}
\usepackage[letterpaper, margin = 2.5cm]{geometry}
\usepackage[T1]{fontenc}
\usepackage{amsmath, amsfonts, amssymb}
\usepackage{hyperref, url} 
\usepackage{fancyhdr}
\usepackage{xcolor}

\newcommand{\R}{\mathbb{R}}


\student{Gabriel Ferreira}   
\semester{Spring 2024}                         
\date{}  

\courselabel{COM S 474/574} 
\exercisesheet{HW1}{Maths Prerequisites}

\school{Department of Computer Science}
\university{Iowa State University}

\begin{document}

%-----------------------------------------------------------------------------------------------
\begin{problem}
This homework covers the prerequisites of COM S 474/574 given for Spring 2024 at Iowa State University. It is due before the Thursday class on Feb 2nd 2024.

\section{Calculus}

\subsection{Numbers}
Give True/False for the following claims:
\begin{enumerate}
    \item True 
    \item False
    \item True
    \item False
    \item True
\end{enumerate}

\subsection{Limits}
\begin{enumerate}
    \item $\lim_{x\rightarrow +\infty} \frac{1}{x} = \textbf{0}$\\\\
    $\frac{1}{10} = 0.1$ \\
    $\frac{1}{100} = 0.01$ \\ 
    $\frac{1}{1000000} = 0.000001$\\

As we can observe, the bigger the denominator is compared to the numerator, the smaller the output is. Therefore, as $x$ approaches $\infty$, the function output approaches 0.\\
    \item $\lim_{x\rightarrow8} \frac{2x^2-17x+8}{8-x} = \textbf{-15}\\\\
    \lim_{x\rightarrow8.1} \frac{2x(8.1)^2 - 17(8.1)+8}{8-8.1} = \frac{1.52}{-.1} = \-15.1\\\\
    \lim_{x\rightarrow8.01} \frac{2x(8.01)^2 - 17(8.1)+8}{8-8.01} = \frac{0.1502}{-0.01} = -15.02$\\

As $x$ approaches 8, the function output approaches -15.\\
    \item $\lim_{x\rightarrow4} \frac{\sqrt{x}-2}{x-4} = \frac{\textbf{1}}{\textbf{4}}\\\\
    \lim_{x\rightarrow4} \frac{(\sqrt{x}-2) \times (\sqrt{x}+2)}{(x-4) \times (\sqrt{x}+2)} = \\\\
    \lim_{x\rightarrow4} \frac{x+2\sqrt{x}-2\sqrt{x}-4}{(x-4) \times (\sqrt{x}+2)} = \\\\
    \lim_{x\rightarrow4} \frac{x-4}{(x-4) \times (\sqrt{x}+2)} = \\\\
    \lim_{x\rightarrow4} \frac{1}{\sqrt{4}+2} = \frac{1}{4}$
\end{enumerate}

\subsection{Integrals}
\noindent Find the following integrals
\begin{enumerate}
    \item $\int (2x^2 + 6x^9) dx =\\\\
    \int (2x^2 + 6x^9) dx = \frac{2x^3}{3} + \frac{6x^{10}}{10} + C \\\\
    \int (2x^2 + 6x^9) dx = .66666667x^3 + .6x^{10} + C$
\end{enumerate}

\subsection{Derivatives}
\noindent Calculate $\frac{df(x)}{dx}$ for the following functions ($f(x)$'s).
\begin{enumerate}
    \item $f(x) = 2x^2 + 6x^9\\\\
    f'(x) = 4x + 54x^8$\\
    \item $f(x) = \sqrt{x^4 + 1} =\\\\
    f'(x) = (x^4+1)^\frac{1}{2}\\\\
    f'(x) = \frac{1}{2}(x^4+1)^{-\frac{1}{2}} \times (4x^3)\\\\
    f'(x) = \frac{4x^3}{2(x^4+1)^{\frac{1}{2}}}\\\\
    f'(x) = \frac{4x^3}{2\sqrt{x^4+1}}\\\\
    f'(x) = \frac{2x^3}{\sqrt{x^4+1}}$\\\\
    \item $f(x) = x \sin{(5x)} =\\\\
    f'(x) = 1\sin(5x)+x\cos(5x)5\\\\
    f'(x) = \sin(x5)+5x\cos(5x)$\\\\
    \item $f(x) = \frac{e^x}{x^2}\\\\
    f'(x) = \frac{e^x x^2 - e^x 2x}{(2x^2)^2}\\\\
    f'(x) = \frac{e^x(x^2-2x)}{x^4}\\\\$
\end{enumerate}
\newpage
\section{Linear Algebra}
\noindent Please include your calculation steps whenever applicable.

\subsection{Notations}
\begin{enumerate}
    \item $\mathbf{I}_3$ = 
    $\begin{bmatrix}1 & 0 & 0 \\ 0 & 1 & 0 \\ 0 & 0 & 1\end{bmatrix}$
    \item Rewrite the equation in the linear algebra form:
    \begin{equation}
        \begin{cases}
        3x + 5y = 1 \\
        2x - y = 0   
        \end{cases} 
    \end{equation}

    $\begin{bmatrix}3 & 5 \\ 2 & -1\end{bmatrix} \begin{pmatrix} y \\ x\end{pmatrix}  = \begin{bmatrix} 1 \\ 0\end{bmatrix} $
    \item Let $A = \mathbf{I}_9$, what is $a_{11}$? What is $a_{51}$?\\\\
    $a_{11}$ is 1 and $a_{51}$ is 0.
    
\end{enumerate}

\subsection{Multiplications}
\noindent Computer the following multiplications (note: not every question is valid):
\begin{enumerate}
    \item $\begin{bmatrix}a & b\end{bmatrix} \begin{bmatrix}c \\ d\end{bmatrix} = ac+bd$
    \item $\begin{bmatrix}a \\ b\end{bmatrix} \begin{bmatrix}c & d\end{bmatrix} = \begin{bmatrix}ac & ad \\ bc & bd\end{bmatrix}$
    \item $\begin{bmatrix}a & b \\ c & d\end{bmatrix} \begin{bmatrix}e & f \\ g & h\end{bmatrix} = \begin{bmatrix}ae+bg & af+bh \\ ce+dg & cf+dh\end{bmatrix}$
    \item $\begin{bmatrix}a & b \\ c & d\end{bmatrix} \begin{bmatrix}e & f\end{bmatrix} =$ Not applicable.
    \item $\begin{bmatrix}a & b \\ c & d\end{bmatrix} \begin{bmatrix}e \\ f\end{bmatrix} = \begin{bmatrix}ae + bf \\ ce + df\end{bmatrix}$
    
\end{enumerate}

\subsection{Other Operations}
\noindent Unless specified otherwise, let $A, B \in \R^{n\times n}$.
\begin{enumerate}
    \item $\mathbf{I}_3^T$ = $\mathbf{I}_3 \in \R^{3\times 3}$
    \item $(AB)^T = B^TA^T$
    \item $(a^TB)^T = B^T(a^T)^T = B^Ta$
    \item $AA^{-1} = A^{-1}A = \mathbf{I}$
    \item Let $A =\begin{bmatrix} 0 & 2 \\ 1 & 3 \end{bmatrix}$, what is $A^{-1}$?\\
    $A^{-1} = \frac{1}{(0 \times 3) - (2 \times 1)} \begin{bmatrix} 3 & -2 \\ -1 & 0 \end{bmatrix}$\\
    $A^{-1} = \frac{1}{-2} \begin{bmatrix} 3 & -2 \\ -1 & 0 \end{bmatrix}$\\
    $A^{-1} = \begin{bmatrix} -\frac{3}{2} & 1 \\ \frac{1}{2} & 0 \end{bmatrix}$
    \item Let $x = \begin{bmatrix} 0 & 2 & 1 & 3 \end{bmatrix}^T$, what is $\lVert x \rVert_2$? What is $\lVert x \rVert_{\infty}$? What is $\lVert x \rVert_1$? What is $\lVert x^T \rVert_2$?\\\\
    $\lVert x \rVert_2 = \sqrt{0^2 + 2^2 + 1^2 + 3^2}$\\
    $\lVert x \rVert_2 = 3.74$\\\\
    $\lVert x \rVert_{\infty} = \max(|0|, |2|, |1|, |3|)$\\
    $\lVert x \rVert_{\infty} = 3$\\\\    
    $\lVert x \rVert_{1} = (0 + 2 + 1 + 3)$\\
    $\lVert x \rVert_{1} = 6$\\\\
    $\lVert x^T \rVert_{2} = \lVert x \rVert_2 $\\
    $\lVert x^T \rVert_{2} = \sqrt{0^2 + 2^2 + 1^2 + 3^2}$\\
    $\lVert x^T \rVert_{2} = 3.74$\\\\
    \item $\lvert \mathbf{I}_{2}\rvert (\text{det}(\mathbf{I}_2)) =$ The determinant of the identity matrix is always 1 \\\\
    \item $\lvert \mathbf{I}_{1024}\rvert = $ The determinant of the identity matrix is always 1 
\end{enumerate}
\newpage
\section{Probability}

\begin{enumerate}
    \item Consider the data set $x = \{1, 2, 3, 4, 5, 6, 77, 88, 999\}$. What is the mean of $x$? What is the median of $x$? What is the standard deviation of $x$?\\\\
    Mean of $x: 131.666667$\\
    Median of $x: 5$\\
    Std deviation of $x: 327.06$
    \item If you were to roll the dice one time, 
    \begin{itemize}
        \item what is the probability it will land on a 1?\\\\
        The probability of landing on a 1 is $\frac{1}{6} = 16.67\%$.\\\\
        \item what is the probability it will not land on a 1?\\\\
        The probability of landing on a 1 is $\frac{5}{6} = 83.33\%$.\\\\
    \end{itemize}
    \item (\textbf{The Monty Hall Problem}) You are presented with three doors, A, B, and C. Behind one door is a valuable prize (like a car), while behind the other two doors, there are less desirable items (like goats). The location of the prize is unknown to you. Suppose you choose one of the three doors (let's say A). Your selection is not opened immediately. Monty, who knows what's behind each door, opens one of the other two doors (let's say C), that reveals a goat (never the car). After Monty reveals a goat behind C, you are given a choice: stick with your original selection (A) or switch to the other unopened door (B). Which decision has a higher chance to win (the car)? Why?\\

    I would switch to the door (C) because this decision would increase my chances of winning the car in 50\%.\\

    With my initial choice, I had $\frac{1}{3}$ of chances to win the desired prize. After knowing the car is not behind door (c), changing my choice to door (b) will increase my chances to $\frac{2}{3}$; while if I would to stick with my initial decision, my chances would remain $\frac{1}{3}$.\\
\end{enumerate}

\section{A ``Bonus" Question (1 pt)}
\noindent From a scale of 1 to 5, how difficult is HW1? 1 is ``I can do it in my sleep". 5 is ``Bowen is ridiculous". 0 is ``I refuse to answer this question". This is for my own reference to improve the quality of future assignments. Thanks!

\textbf{3 out of 5.}
    
\end{problem}

\end{document}